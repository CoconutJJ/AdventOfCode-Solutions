\documentclass{article}
\usepackage{geometry}
\usepackage{amsmath}
\usepackage{amssymb}
\begin{document}
\setlength{\parindent}{0ex}
Define the linear transformation \(T\) as follows,

\begin{equation}
    T(a,b,c,d,e,f,g,h,i) = [i, a, b + i, c, d, e, f, g, h]
\end{equation}

We can see that \(T\) is a transition function that maps the current day's state
to the next day's state. \(a\) is the number of fish with internal timer 8,
\(b\) is the number of fish with internal timer 7 and so on...

The matrix representation of \(T\) is,
\begin{equation}
    A_T =
    \begin{bmatrix}
        0 & 0 & 0 & 0 & 0 & 0 & 0 & 0 & 1 \\
        1 & 0 & 0 & 0 & 0 & 0 & 0 & 0 & 0 \\
        0 & 1 & 0 & 0 & 0 & 0 & 0 & 0 & 1 \\
        0 & 0 & 1 & 0 & 0 & 0 & 0 & 0 & 0 \\
        0 & 0 & 0 & 1 & 0 & 0 & 0 & 0 & 0 \\
        0 & 0 & 0 & 0 & 1 & 0 & 0 & 0 & 0 \\
        0 & 0 & 0 & 0 & 0 & 1 & 0 & 0 & 0 \\
        0 & 0 & 0 & 0 & 0 & 0 & 1 & 0 & 0 \\
        0 & 0 & 0 & 0 & 0 & 0 & 0 & 1 & 0 \\
    \end{bmatrix}
\end{equation}

Let \(\vec{v_1}\) be the state on day 1. We wish to compute \(\vec{v_{256}}\),
which can be expressed as

\begin{equation}
    \vec{v_{256}} = (A_T)^{256}\vec{v_1}
\end{equation}

Computing \((A_T)^{256}\) is incredibly expensive. Let us diagonalize \(A_T\) by
computing it's eigenvectors



\begin{align}
     & A_T\vec{v} = \lambda v    \\
     & \det(A_T - \lambda I) = 0
\end{align}

\begin{equation}
    \det
    \begin{bmatrix}
        -\lambda & 0        & 0        & 0        & 0        & 0        & 0        & 0        & 1        \\
        1        & -\lambda & 0        & 0        & 0        & 0        & 0        & 0        & 0        \\
        0        & 1        & -\lambda & 0        & 0        & 0        & 0        & 0        & 1        \\
        0        & 0        & 1        & -\lambda & 0        & 0        & 0        & 0        & 0        \\
        0        & 0        & 0        & 1        & -\lambda & 0        & 0        & 0        & 0        \\
        0        & 0        & 0        & 0        & 1        & -\lambda & 0        & 0        & 0        \\
        0        & 0        & 0        & 0        & 0        & 1        & -\lambda & 0        & 0        \\
        0        & 0        & 0        & 0        & 0        & 0        & 1        & -\lambda & 0        \\
        0        & 0        & 0        & 0        & 0        & 0        & 0        & 1        & -\lambda \\
    \end{bmatrix}
    = -\lambda^9 + \lambda^2 + 1 = 0
\end{equation}

You get the point, I'm not solving a nonic-polynomial...

\end{document}