\documentclass{article}
\usepackage[margin=0.5in]{geometry}
\usepackage{amsmath}
\usepackage{amssymb}
\title{Advent of Code: Day 6 Solution}
\date{}
\begin{document}
    \maketitle
    \noindent Each fish reproduces at the end of their own internal timer, and resets
    its own timer to 6.\\
    
    \noindent Define \(T(L,D)\) be to the number of fish after \(D\) days
    starting with a single fish having an internal timer of \(L\). We have three
    cases
    \begin{enumerate}
        \item [Case 1.] \(D = 0, L \in \mathbb{N}\). There are no more days remaining,
        this fish cannot reproduce. We still have only 1 fish.
        \item [Case 2.] \(L = 0, D > 0\). \(D\) is at minimum 1, so this fish can reproduce.
        This gives us another fish with a timer set to 8 and the original fish's timer is 
        reset to 6. Thus in \(D - 1\) days we will have \(T(8, D-1) + T(6, D - 1)\)
        \item [Case 3.] The fish is not yet ready to reproduce, we skip until
        either the fish timer expires again or the days have been exhausted -
        whichever comes first. This is \(T(L - \min(L,D), D - \min(L,D))\)
    \end{enumerate}
    We get the following recurrence,
    \begin{eqnarray}
        T(L,D) = \begin{cases}
            1 & \textrm{if } D = 0, L \in \mathbb{N}\\
            T(8, D - 1) + T(6, D - 1) & \textrm{if } L = 0, D > 0\\
            T(L - \min(L,D), D - \min(L,D)) & \textrm{otherwise}
        \end{cases}
    \end{eqnarray}
    \begin{enumerate}
        \item [Base Case.] \(k = 0, 0 \leq L \leq 8\).\\ 
        Then \(T(L,k) = T(L,0) = 1\) as wanted.
        \item[I.H] Suppose by induction that
        \begin{equation}
            P(n): T(L,n) \textrm{ gives the number of fish after } n \textrm{ days for any } 0 \leq L \leq 8
        \end{equation}
        holds whenever \(0 \leq n < k\).
        \item [W.T.S] \(P(k)\) holds.
        \(T(L,k)\) has 3 cases,
        \begin{enumerate}
            \item \(k = 0\). Then \(T(L,k) = T(L,0) = 1\) as wanted (by base case).
            \item \(L = 0, k > 0\). \(T(L,k) = T(8, k -1) + T(6, k - 1)\).\\
            Since \(k - 1 < k\), by I.H, \(T(8, k - 1)\) returns the correct value.\\
            Since \(k - 1 < k\), by I.H, \(T(6, k - 1)\) returns the correct value.\\
            It is easy to see that number of fish after \(k\) days will the be
            the sum of these two values. Therefore \(T(L,k) = T(8, k -1) + T(6, k - 1)\) as wanted.
            \item \(L > 0, k > 0\). Then we have 3 cases,
            \begin{enumerate}
                \item[\(L < k\):] Then it follows that 
                \begin{align}
                    T(L,k) &= T(L - \min(L,k), k - \min(L,k))\\ 
                    &=T(0,k') \textrm{ where } k' = k - L
                \end{align}
                Since \(k' = k - L < k\), \(T(0,k')\) returns the correct value. (By I.H)
                \item[\(L > k\):] Then it follows that 
                \begin{align}
                    T(L,k) &= T(L - \min(L,k), k - \min(L,k))\\ 
                    &=T(L',0) \textrm{ where } L' = L - k
                \end{align}
                Since \(0 < k\), \(T(L',0)\) returns the correct value. (By I.H)
                \item[\(L = k\):] Then it follows that 
                \begin{align}
                    T(L,k) &= T(L - \min(L,k), k - \min(L,k))\\ 
                    &=T(0,0)\\
                    & = 1
                \end{align}
                Since \(0 < k\), \(T(0,0)\) returns the correct value. (By I.H)

            \end{enumerate}

        \end{enumerate}
    \end{enumerate}
\end{document}